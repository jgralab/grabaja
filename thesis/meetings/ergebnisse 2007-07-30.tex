% Diese Vorlage wurde von Simon Berwert erstellt. Weitere Erkl�rungen findest du auf folgender Seite: http://www.unimac.ch/students/latex.de.html



% A. PR�AMBEL
% ***************************************************************************************************

\documentclass[smallheadings,headsepline, titlepage,12pt,a4paper]{scrartcl}
% Hier gibt man an, welche Art von Dokument man schreiben m�chte.
% Möglichkeiten in {}: scrartcl, scrreprt, scrbook, aber auch: article, report, book
\usepackage[ngerman]{babel} % erm�glicht deutsche Silbentrennung und direkte Eingabe von Umlauten, ...
\usepackage{ucs}
\usepackage[ansinew]{inputenc} % teilt LaTeX die Texcodierung mit. Bei Windowssystemen: ansinew
\usepackage[T1]{fontenc} % erm�glicht die Silbentrennung von W�rtern mit Umlauten
\usepackage{hyperref} % PDF wird mit Lesezeichen (verlinktes Inhaltsverzeichnis) versehen (bei Betrachtung mit Acrobat Reader sichtbar)
\typearea{12} % Breite des bedruckten Bereiches vergr�ssern (funktioniert nur in \documentclass mit: scrreprt, scrartcl, scrbook)
\pagestyle{headings} % schaltet Kopfzeilen ein
\clubpenalty = 10000 % schliesst Schusterjungen aus
\widowpenalty = 10000 % schliesst Hurenkinder aus

\usepackage{longtable} % erm�glicht die Verwendung von langen Tabellen
\usepackage{graphicx} % erm�glicht die Verwendung von Graphiken.
\usepackage{times}
\begin{document}

% B. TITELSEITE UND INHALTSVERZEICHNIS
% ***************************************************************************************************

\titlehead{Universit�t Koblenz-Landau\\
Institut f�r Softwaretechnik\\
Universit�tsstr. 1\\
56072 Koblenz}

\subject{Studienarbeit Java-Faktenextraktor f�r GUPRO}
\title{Ergebnisse des Treffens vom 30. Juli 2007}
\author{Arne Baldauf \url{abaldauf@uni-koblenz.de}\\ Nicolas Vika \url{ultbreit@uni-koblenz.de}}
\date{\today}
\maketitle
\newpage

%\tableofcontents
% Dieser Befehl erstellt das Inhaltsverzeichnis. Damit die Seitenzahlen korrekt sind, muss das Dokument zweimal gesetzt werden!
%\newpage

% C. DOKUMENTHISTORIE
% ***************************************************************************************************
%\begin{table}
%	\begin{center}
%	\begin{tabular}{|l|l|l|l|l|}
%	  \hline
%	  Version & Status & Datum & Autor(en) & Erl�uterung \\
%	  \hline \hline
%		1.0 & WIP & 01.06.2007 & Arne Baldauf & initiale Version \\ \hline
%	\end{tabular}
%	\end{center}
%\end{table}

% D. HAUPTTEIL
% ***************************************************************************************************
\begin{itemize}
	\item{Der Text zum Extraktor soll wie folgt �berarbeitet werden:}
	\begin{itemize}
		\item{Fortsetzung gem�� dem Stand der Implementation.}
		\item{Noch offene Punkte der letzten Besprechung gelten weiterhin.}
		\item{Es sollen allgemein kurze Beschreibungen zu den Teilen hinzugef�gt werden, welche nicht beim Leser vorausgesetzt werden k�nnen (z.B. zu Schemabeschreibungssprache).}
		\item{Es soll allgemein etwas mehr mit Hervorhebungen gearbeitet werden.}
		\item{Im Ablaufdiagramm soll die Parameterauswertung hinzugef�gt werden.}
		\item{Im Ablaufdiagram sollen Kontroll- und Datenfluss deutlicher von einander getrennt werden.}
		\item{Allgemein soll die Konsistenz zwischen Text und Abbildungen (z.B. bei Bezeichnern) verbessert werden.}
		\item{In Ausschnitten aus dem Schema sollen auch die Schl�sselworte der Schemasprache hervorgehoben werden (selbstdefiniertes Codehighlighting).}
		\item{Im Textteil zum Schema soll kurz auf die Ableitung der Kantenklassen eingegangen werden.}
		\item{Weitere Punkte siehe Notizen und Anmerkungen in der ausgedruckten Fassung.}
	\end{itemize}
	\item{Bei Imports soll mittels eigener Typen zwischen einem Typimport (z.B. \texttt{"x.y.z"}) und einem Packageimport (z.B. \texttt{"x.y.*"}) unterschieden werden.}
	\item{Die Absch�tzung der Knoten- / Kantenzahl soll entfallen.}
  \item{Das n�chste Treffen findet am 13. August 2007 um 11:00 s.t. in B108 statt.}
  \item{Einreichung der Dokumente bis 10. August 2007 (morgens).}
  \end{itemize}
\end{document}