% Diese Vorlage wurde von Simon Berwert erstellt. Weitere Erkl�rungen findest du auf folgender Seite: http://www.unimac.ch/students/latex.de.html



% A. PR�AMBEL
% ***************************************************************************************************

\documentclass[smallheadings,headsepline, titlepage,12pt,a4paper]{scrartcl}
% Hier gibt man an, welche Art von Dokument man schreiben m�chte.
% Möglichkeiten in {}: scrartcl, scrreprt, scrbook, aber auch: article, report, book
\usepackage[ngerman]{babel} % erm�glicht deutsche Silbentrennung und direkte Eingabe von Umlauten, ...
\usepackage{ucs}
\usepackage[ansinew]{inputenc} % teilt LaTeX die Texcodierung mit. Bei Windowssystemen: ansinew
\usepackage[T1]{fontenc} % erm�glicht die Silbentrennung von W�rtern mit Umlauten
\usepackage{hyperref} % PDF wird mit Lesezeichen (verlinktes Inhaltsverzeichnis) versehen (bei Betrachtung mit Acrobat Reader sichtbar)
\typearea{12} % Breite des bedruckten Bereiches vergr�ssern (funktioniert nur in \documentclass mit: scrreprt, scrartcl, scrbook)
\pagestyle{headings} % schaltet Kopfzeilen ein
\clubpenalty = 10000 % schliesst Schusterjungen aus
\widowpenalty = 10000 % schliesst Hurenkinder aus

\usepackage{longtable} % erm�glicht die Verwendung von langen Tabellen
\usepackage{graphicx} % erm�glicht die Verwendung von Graphiken.
\usepackage{times}
\begin{document}

% B. TITELSEITE UND INHALTSVERZEICHNIS
% ***************************************************************************************************

\titlehead{Universit�t Koblenz-Landau\\
Institut f�r Softwaretechnik\\
Universit�tsstr. 1\\
56072 Koblenz}

\subject{Studienarbeit Java-Faktenextraktor f�r GUPRO}
\title{Ergebnisse des Treffens vom 27. M�rz 2006}
\author{Arne Baldauf \url{abaldauf@uni-koblenz.de}\\ Nicolas Vika \url{ultbreit@uni-koblenz.de}}
\date{\today}
\maketitle
\newpage

%\tableofcontents
% Dieser Befehl erstellt das Inhaltsverzeichnis. Damit die Seitenzahlen korrekt sind, muss das Dokument zweimal gesetzt werden!
\newpage

% C. DOKUMENTHISTORIE
% ***************************************************************************************************
\begin{table}
	\begin{center}
	\begin{tabular}{|l|l|l|l|l|}
	  \hline
	  Version & Status & Datum & Autor(en) & Erl�uterung \\
	  \hline \hline
		1.0 & WIP & 31.03.2006 & Arne Baldauf & initiale Version \\ \hline
	\end{tabular}
	\end{center}
\end{table}

% D. HAUPTTEIL
% ***************************************************************************************************
\begin{itemize}
	\item{Es sollen folgende �nderungen am Metamodell durchgef�hrt werden:}
	\begin{itemize}
		\item{Die beiden Elemente ArrayInitializer und ArrayCreation sollen von einander getrennt werden.}
		\item{Es sollen TypeParameterDeclaration-Elemente in Konstruktordefinitionen m�glich sein.}
		\item{Falls sich dies nach einer �berpr�fung als m�glich erweist, sollen Deklarationen im ForIterator-Element erm�glicht werden.}
		\item{Es soll eine InstanceOf-Expression eingef�hrt werden.}
		\item{RAW-Typen sollen mittels eines Flags identifiziert werden.}
		\item{Die Definitionen von Class, Interface und Enum sollen soweit wie m�glich unter der Definition zusammengefasst werden.}
		\item{Die VariableLengthParameterDeclaration soll von der normalen ParameterDeclaration abgeleitet werden.}
		\item{Alle formalen Parameter sollen "`Parameter"' im Namen tragen.}
		\item{Alle aktuellen Parameter sollen "`Argument"' im Namen tragen.}
		\item{Der Bezeichner f�r das Element des Methodenaufrufs ("`MethodCall"' oder "`MethodInvocation"') soll der Java Language Specification entsprechend benannt werden.}
		\item{Generics-Elemente sollen dadurch identifiziert werden, dass sie eine Kante "`isTypeSpecOf"' auf das entsprechende "`Ident"'-Element besitzen.}
		\item{Das "`Ident"'-Element soll zu "`Indetifier"' umbenannt werden.}
		\item{Alle Methoden sollen als "`MethodDefinition"' bezeichnet werden, nur wenn diese keinen Rumpf besitzen, soll "`MethodDeclaration"' gew�hlt werden.}
		\item{Beim Methodenaufruf soll nicht zwischen normalen Aufrufen und Aufrufen von Methoden in der Superklasse unterschieden werden.}
		\item{Die Struktur zur Repr�sentation der Arraygrenzen ("`Bounds"') soll vereinfacht werden.}
	\end{itemize}
  \item{Das n�chste Treffen findet nach Vereinbarung statt.}
  \end{itemize}
\end{document}