% Diese Vorlage wurde von Simon Berwert erstellt. Weitere Erkl�rungen findest du auf folgender Seite: http://www.unimac.ch/students/latex.de.html



% A. PR�AMBEL
% ***************************************************************************************************

\documentclass[smallheadings,headsepline, titlepage,12pt,a4paper]{scrartcl}
% Hier gibt man an, welche Art von Dokument man schreiben m�chte.
% Möglichkeiten in {}: scrartcl, scrreprt, scrbook, aber auch: article, report, book
\usepackage[ngerman]{babel} % erm�glicht deutsche Silbentrennung und direkte Eingabe von Umlauten, ...
\usepackage{ucs}
\usepackage[ansinew]{inputenc} % teilt LaTeX die Texcodierung mit. Bei Windowssystemen: ansinew
\usepackage[T1]{fontenc} % erm�glicht die Silbentrennung von Wörtern mit Umlauten
\usepackage{hyperref} % PDF wird mit Lesezeichen (verlinktes Inhaltsverzeichnis) versehen (bei Betrachtung mit Acrobat Reader sichtbar)
\typearea{12} % Breite des bedruckten Bereiches vergr�ssern (funktioniert nur in \documentclass mit: scrreprt, scrartcl, scrbook)
\pagestyle{headings} % schaltet Kopfzeilen ein
\clubpenalty = 10000 % schliesst Schusterjungen aus
\widowpenalty = 10000 % schliesst Hurenkinder aus

\usepackage{longtable} % erm�glicht die Verwendung von langen Tabellen
\usepackage{graphicx} % erm�glicht die Verwendung von Graphiken.
\usepackage{times}
\begin{document}

% B. TITELSEITE UND INHALTSVERZEICHNIS
% ***************************************************************************************************

\titlehead{Universit�t Koblenz-Landau\\
Institut f�r Softwaretechnik\\
Universit�tsstr. 1\\
56072 Koblenz}

\subject{Studienarbeit Java-Faktenextraktor f�r GUPRO}
\title{Ergebnisse des Treffens vom 06. Februar 2006}
\author{Arne Baldauf \url{abaldauf@uni-koblenz.de}\\ Nicolas Vika \url{ultbreit@uni-koblenz.de}}
\date{\today}
\maketitle
\newpage

%\tableofcontents
% Dieser Befehl erstellt das Inhaltsverzeichnis. Damit die Seitenzahlen korrekt sind, muss das Dokument zweimal gesetzt werden!
\newpage

% C. DOKUMENTHISTORIE
% ***************************************************************************************************
\begin{table}
	\begin{center}
	\begin{tabular}{|l|l|l|l|l|}
	  \hline
	  Version & Status & Datum & Autor(en) & Erl�uterung \\
	  \hline \hline
		1.0 & WIP & 11.02.2006 & Arne Baldauf & initiale Version \\ \hline
	\end{tabular}
	\end{center}
\end{table}

% D. HAUPTTEIL
% ***************************************************************************************************
\begin{itemize}
  \item{Das Dokument zur Online-Recherche soll �berarbeitet werden, insbesondere soll eine gemeinsame Bewertungsgrundlage der Tools ersichtlich werden, Ausschlusskriterien sollen explizit aufgef�hrt werden}
  \item{Das Dokument zur Online-Recherche (und weitere Dokumente, welche sp�ter in die finale Ausarbeitung �bernommen werden) soll hinsichtlich einer gleichf�rmigen Gliederung und einer gleichf�rmigen Vorgehensweise �berarbeitet werden. Beschreibende und bewertende Teile des Textes sollen dabei getrennt als solche erkennbar sein}
  \item{Die Ergebnisse der genauen Untersuchung des Werkzeuges ANTLR sollen in einem eigenen Dokument festgehalten werden}
  \item{Beim Erstellen der Dokumente soll verst�rkt auf eine wissenschaftliche Schreibweise und Vorgehensweise geachtet werden}
  \item{Ein Zwischentreffen mit Dr. Volker Riediger findet am 13. Februar 2006 um 13 Uhr in MB121 statt}
  \item{Das n�chste Treffen findet am 20. Februar 2006 um 13 Uhr in MB108 statt}
  \end{itemize}
\end{document}