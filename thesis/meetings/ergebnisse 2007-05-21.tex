% Diese Vorlage wurde von Simon Berwert erstellt. Weitere Erkl�rungen findest du auf folgender Seite: http://www.unimac.ch/students/latex.de.html



% A. PR�AMBEL
% ***************************************************************************************************

\documentclass[smallheadings,headsepline, titlepage,12pt,a4paper]{scrartcl}
% Hier gibt man an, welche Art von Dokument man schreiben m�chte.
% Möglichkeiten in {}: scrartcl, scrreprt, scrbook, aber auch: article, report, book
\usepackage[ngerman]{babel} % erm�glicht deutsche Silbentrennung und direkte Eingabe von Umlauten, ...
\usepackage{ucs}
\usepackage[ansinew]{inputenc} % teilt LaTeX die Texcodierung mit. Bei Windowssystemen: ansinew
\usepackage[T1]{fontenc} % erm�glicht die Silbentrennung von W�rtern mit Umlauten
\usepackage{hyperref} % PDF wird mit Lesezeichen (verlinktes Inhaltsverzeichnis) versehen (bei Betrachtung mit Acrobat Reader sichtbar)
\typearea{12} % Breite des bedruckten Bereiches vergr�ssern (funktioniert nur in \documentclass mit: scrreprt, scrartcl, scrbook)
\pagestyle{headings} % schaltet Kopfzeilen ein
\clubpenalty = 10000 % schliesst Schusterjungen aus
\widowpenalty = 10000 % schliesst Hurenkinder aus

\usepackage{longtable} % erm�glicht die Verwendung von langen Tabellen
\usepackage{graphicx} % erm�glicht die Verwendung von Graphiken.
\usepackage{times}
\begin{document}

% B. TITELSEITE UND INHALTSVERZEICHNIS
% ***************************************************************************************************

\titlehead{Universit�t Koblenz-Landau\\
Institut f�r Softwaretechnik\\
Universit�tsstr. 1\\
56072 Koblenz}

\subject{Studienarbeit Java-Faktenextraktor f�r GUPRO}
\title{Ergebnisse des Treffens vom 21. Mai 2007}
\author{Arne Baldauf \url{abaldauf@uni-koblenz.de}\\ Nicolas Vika \url{ultbreit@uni-koblenz.de}}
\date{\today}
\maketitle
\newpage

%\tableofcontents
% Dieser Befehl erstellt das Inhaltsverzeichnis. Damit die Seitenzahlen korrekt sind, muss das Dokument zweimal gesetzt werden!
%\newpage

% C. DOKUMENTHISTORIE
% ***************************************************************************************************
%\begin{table}
%	\begin{center}
%	\begin{tabular}{|l|l|l|l|l|}
%	  \hline
%	  Version & Status & Datum & Autor(en) & Erl�uterung \\
%	  \hline \hline
%		1.0 & WIP & 01.06.2007 & Arne Baldauf & initiale Version \\ \hline
%	\end{tabular}
%	\end{center}
%\end{table}

% D. HAUPTTEIL
% ***************************************************************************************************
\begin{itemize}
	\item{Bei der Implementation des Typechecking sollen folgende Dinge ber�cksichtigt werden:}
	\begin{itemize}
		\item{Bei der Auswertung von "`import"'-Klauseln sollen die Package- / Paketnamen als voll qualifizierter Name (fullqualifiedname) behandelt werden.}
		\item{Identifier sollen pro Namespace erkannt / unterschieden werden.}
		\item{Evtl. schon vorbereitend f�r das Merging: Falls m�glich, soll bei Typangaben die Kante direkt auf die entspr. Deklaration zeigen. Im Fall eines QualifiedType auch f�r die entspr. Oberklassen / Packages.}
	\end{itemize}
	\item{Bei der Konvertierung des von ANTLR generierten AST in einen TGraphen sollen folgende Dinge ge�ndert werden:}
	\begin{itemize}
		\item{Bei konstanten Werten einfacher Typen im Quelltext (Ganz- und Flie�kommazahlenwerte, char-Werte, boolean-Werte) soll neben dem numerischen / logischen Werte auch noch der zugeh�rige String gespeichert werden (da insbesondere numerische Werte in verschiedenen Schreibweisen dargestellt werden k�nnen).}
		\item{Konstante Ganzzahlwerte sollen immer als long gespeichert werden (da eine Unterscheidung zwischen den unterschiedlichen Typen oft nicht einfach so m�glich ist).}
	\end{itemize}
	\item{Das Metamodell soll bis 25.05.2007 wie folgt �berarbeitet werden:}
	\begin{itemize}
		\item{Unterteilung des Modells in sinnvolle Packages.}
		\item{Verschieben der entspr. Klassen in die einzelnen Packages.}
		\item{Unterteilung des Modelldiagrammes in die f�r die Packages entspr. Teilelemente.}
		\item{Erstellung eines �bersichtsdiagrammes, welches die Beziehungen der einzelnen Pakete darstellt.}
		\item{Nach Abschluss dieser Arbeiten sollen m�glichst keine �nderungen am Metamodell mehr vorgenommen werden.}
	\end{itemize}
  \item{Das n�chste Treffen findet am 11. Juni 2007 um 16:00s.t. in B108 statt.}
  \item{Einreichung der Dokumente bis 08. Juni 2007 (abends).}
  \end{itemize}
\end{document}