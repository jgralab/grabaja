% Diese Vorlage wurde von Simon Berwert erstellt. Weitere Erkl�rungen findest du auf folgender Seite: http://www.unimac.ch/students/latex.de.html



% A. PR�AMBEL
% ***************************************************************************************************

\documentclass[smallheadings,headsepline, titlepage,12pt,a4paper]{scrartcl}
% Hier gibt man an, welche Art von Dokument man schreiben m�chte.
% Möglichkeiten in {}: scrartcl, scrreprt, scrbook, aber auch: article, report, book
\usepackage[ngerman]{babel} % erm�glicht deutsche Silbentrennung und direkte Eingabe von Umlauten, ...
\usepackage{ucs}
\usepackage[ansinew]{inputenc} % teilt LaTeX die Texcodierung mit. Bei Windowssystemen: ansinew
\usepackage[T1]{fontenc} % erm�glicht die Silbentrennung von W�rtern mit Umlauten
\usepackage{hyperref} % PDF wird mit Lesezeichen (verlinktes Inhaltsverzeichnis) versehen (bei Betrachtung mit Acrobat Reader sichtbar)
\typearea{12} % Breite des bedruckten Bereiches vergr�ssern (funktioniert nur in \documentclass mit: scrreprt, scrartcl, scrbook)
\pagestyle{headings} % schaltet Kopfzeilen ein
\clubpenalty = 10000 % schliesst Schusterjungen aus
\widowpenalty = 10000 % schliesst Hurenkinder aus

\usepackage{longtable} % erm�glicht die Verwendung von langen Tabellen
\usepackage{graphicx} % erm�glicht die Verwendung von Graphiken.
\usepackage{times}
\begin{document}

% B. TITELSEITE UND INHALTSVERZEICHNIS
% ***************************************************************************************************

\titlehead{Universit�t Koblenz-Landau\\
Institut f�r Softwaretechnik\\
Universit�tsstr. 1\\
56072 Koblenz}

\subject{Studienarbeit Java-Faktenextraktor f�r GUPRO}
\title{Ergebnisse des Treffens vom 11. Juni 2007}
\author{Arne Baldauf \url{abaldauf@uni-koblenz.de}\\ Nicolas Vika \url{ultbreit@uni-koblenz.de}}
\date{\today}
\maketitle
\newpage

%\tableofcontents
% Dieser Befehl erstellt das Inhaltsverzeichnis. Damit die Seitenzahlen korrekt sind, muss das Dokument zweimal gesetzt werden!
%\newpage

% C. DOKUMENTHISTORIE
% ***************************************************************************************************
%\begin{table}
%	\begin{center}
%	\begin{tabular}{|l|l|l|l|l|}
%	  \hline
%	  Version & Status & Datum & Autor(en) & Erl�uterung \\
%	  \hline \hline
%		1.0 & WIP & 01.06.2007 & Arne Baldauf & initiale Version \\ \hline
%	\end{tabular}
%	\end{center}
%\end{table}

% D. HAUPTTEIL
% ***************************************************************************************************
\begin{itemize}
	\item{Das Metamodell soll wie folgt �berarbeitet werden:}
	\begin{itemize}
		\item{Es soll in Zukunft die bereinigte Fassung von Volker Riedieger im Enterprise Architect-Format verwendet und aktualisiert werden.}
		\item{Bei Typspezifikationen sollen diese nicht mehr auf den Identifier der Definition zeigen, sondern direkt auf die Definition.}
		\item{R�mpfe von Methoden und Konstruktoren sollen nicht mehr als Compound behandelt werden, sondern als Block.}
		\item{QualifiedType soll in Zukunft von QualifiedName und TypeSpecification abgeleitet werden.}
		\item{An JavaPackage soll nicht mehr direkt der Identifier angehangen werden, sondern ein QualifiedName.}
	\end{itemize}
	\item{Bei der Konvertierung des von ANTLR generierten AST in einen TGraphen sollen folgende Dinge ge�ndert werden:}
	\begin{itemize}
		\item{Bei Unterelementen, wie z. B. der else-Teil eines If, sollen eventuell vorhandene Schl�sselworte nicht Bestandteil der Positions- und L�ngenangaben der entsprechenden Kante sein.}
		\item{Annotationen, die Elemente beschreiben, m�ssen mit in die Positions- und L�ngenangaben des Gesamtelements einfliessen.}
		\item{Zu jeder Typdefinition sollte der vollqualifizierte Name gespeichert werden, damit das Typechecking einfacher wird.}
		\item{Der Identifier einer Klassendefinition und der Identifier des Konstruktors der Klasse, sollen nicht verschmolzen werden.}
	\end{itemize}
	\item{Eine Geschwindigkeitsoptimierung des Javaextraktors soll erst dann stattfinden, wenn dieser gravierend langsam  werden sollte (Extraktionsprozess ben�tigt mehrere Tage).}
  \item{Das n�chste Treffen findet (nach Verschiebung) am 02. Juli 2007 um 16:00 s.t. in B108 statt.}
  \item{Einreichung der Dokumente (nach Verschiebung) bis 29. Juni 2007 (morgens).}
  \end{itemize}
\end{document}