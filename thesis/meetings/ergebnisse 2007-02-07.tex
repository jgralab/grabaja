% Diese Vorlage wurde von Simon Berwert erstellt. Weitere Erkl�rungen findest du auf folgender Seite: http://www.unimac.ch/students/latex.de.html



% A. PR�AMBEL
% ***************************************************************************************************

\documentclass[smallheadings,headsepline, titlepage,12pt,a4paper]{scrartcl}
% Hier gibt man an, welche Art von Dokument man schreiben m�chte.
% Möglichkeiten in {}: scrartcl, scrreprt, scrbook, aber auch: article, report, book
\usepackage[ngerman]{babel} % erm�glicht deutsche Silbentrennung und direkte Eingabe von Umlauten, ...
\usepackage{ucs}
\usepackage[ansinew]{inputenc} % teilt LaTeX die Texcodierung mit. Bei Windowssystemen: ansinew
\usepackage[T1]{fontenc} % erm�glicht die Silbentrennung von W�rtern mit Umlauten
\usepackage{hyperref} % PDF wird mit Lesezeichen (verlinktes Inhaltsverzeichnis) versehen (bei Betrachtung mit Acrobat Reader sichtbar)
\typearea{12} % Breite des bedruckten Bereiches vergr�ssern (funktioniert nur in \documentclass mit: scrreprt, scrartcl, scrbook)
\pagestyle{headings} % schaltet Kopfzeilen ein
\clubpenalty = 10000 % schliesst Schusterjungen aus
\widowpenalty = 10000 % schliesst Hurenkinder aus

\usepackage{longtable} % erm�glicht die Verwendung von langen Tabellen
\usepackage{graphicx} % erm�glicht die Verwendung von Graphiken.
\usepackage{times}
\begin{document}

% B. TITELSEITE UND INHALTSVERZEICHNIS
% ***************************************************************************************************

\titlehead{Universit�t Koblenz-Landau\\
Institut f�r Softwaretechnik\\
Universit�tsstr. 1\\
56072 Koblenz}

\subject{Studienarbeit Java-Faktenextraktor f�r GUPRO}
\title{Ergebnisse des Treffens vom 07. Februar 2007}
\author{Arne Baldauf \url{abaldauf@uni-koblenz.de}\\ Nicolas Vika \url{ultbreit@uni-koblenz.de}}
\date{\today}
\maketitle
\newpage

%\tableofcontents
% Dieser Befehl erstellt das Inhaltsverzeichnis. Damit die Seitenzahlen korrekt sind, muss das Dokument zweimal gesetzt werden!
%\newpage

% C. DOKUMENTHISTORIE
% ***************************************************************************************************
%\begin{table}
%	\begin{center}
%	\begin{tabular}{|l|l|l|l|l|}
%	  \hline
%	  Version & Status & Datum & Autor(en) & Erl�uterung \\
%	  \hline \hline
%		1.0 & WIP & 17.02.2007 & Arne Baldauf & initiale Version \\ \hline
%	\end{tabular}
%	\end{center}
%\end{table}

% D. HAUPTTEIL
% ***************************************************************************************************
\begin{itemize}
	\item{F�r die Verifikation der Extraktorfunktionalit�t sollen folgende Techniken in Betracht gezogen werden:}
	\begin{itemize}
		\item{Erzeugen einer graph. Darstellung mittels GraphViz / TG2DOT (f�r Baumstrukturen komplett, f�r den finalen TGraphen nur f�r ausgew�hlte Teilbereiche).}
		\item{evtl.: Erzeugen von Quelltext aus dem Graphen und Vergleich mit dem Original.}
	\end{itemize}
	\item{Bisher mehrfach vorkommende Kantennamen (dies ist in JGraLab so nicht m�glich) sollen durch spezielle Namen ersetzt werden, aber von einem abstrakten (generischen, gemeinsamen) Kantentyp mit der urspr�nglichen Bezeichnung abgeleitet werden.}
	\item{Folgende kleinere �nderungen sollen an Metamodell und Schema durchgef�hrt werden:}
	\begin{itemize}
		\item{"`TypeParameterAccess"' soll in "`TypeParameterUsage"' umbenannt werden.}
		\item{Der allgemeine Kantentyp soll von "`EdgeWithPositionInformations"' zu "`AttributedEdge"' umbenannt werden.}
		\item{Kantennamen und Attribute im systembezogenen Teil des Metamodells ("`Kopf"') sollen mit denen aus dem existierenden C-Extraktor �bereinstimmen.}
		\item{Diverse kleinere Fehlerkorrekturen wie besprochen.}
	\end{itemize}
	\item{Mit Kommentaren aus dem Quelltext soll wie folgt verfahren werden:}
	\begin{itemize}
		\item{Aufsammeln der Kommentaren durch den Lexer.}
		\item{Evtl. ben�tigte neue Exceptions im Lexer m�ssen von den bereits existierenden abgeleitet werden (da keine neuen vorkommen d�rfen).}
		\item{Die Kommentare sollen in eine Collection �bernommen werden, welche zur TranslationUnit geh�rt.}
		\item{Falls dies einfach m�glich ist, sollen JavaDoc-Kommentare an das jeweilige Element gebunden werden.}
		\item{Kantennamen und Attribute im systembezogenen Teil des Metamodells ("`Kopf"') sollen mit denen aus dem existierenden C-Extraktor �bereinstimmen.}
		\item{Diverse kleinere Fehlerkorrekturen wie besprochen.}
	\end{itemize}
  \item{Das n�chste Treffen findet am 22. Februar 2007 um 15:00s.t. in B108 statt.}
  \item{Einreichung der Dokumente bis 20. Februar 2007 (abends).}
  \end{itemize}
\end{document}