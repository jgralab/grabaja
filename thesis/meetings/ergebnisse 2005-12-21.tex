% Diese Vorlage wurde von Simon Berwert erstellt. Weitere Erkl�rungen findest du auf folgender Seite: http://www.unimac.ch/students/latex.de.html



% A. PR�AMBEL
% ***************************************************************************************************

\documentclass[smallheadings,headsepline, titlepage,12pt,a4paper]{scrartcl}
% Hier gibt man an, welche Art von Dokument man schreiben m�chte.
% Möglichkeiten in {}: scrartcl, scrreprt, scrbook, aber auch: article, report, book
\usepackage[ngerman]{babel} % erm�glicht deutsche Silbentrennung und direkte Eingabe von Umlauten, ...
\usepackage{ucs}
\usepackage[ansinew]{inputenc} % teilt LaTeX die Texcodierung mit. Bei Windowssystemen: ansinew
\usepackage[T1]{fontenc} % erm�glicht die Silbentrennung von Wörtern mit Umlauten
\usepackage{hyperref} % PDF wird mit Lesezeichen (verlinktes Inhaltsverzeichnis) versehen (bei Betrachtung mit Acrobat Reader sichtbar)
\typearea{12} % Breite des bedruckten Bereiches vergr�ssern (funktioniert nur in \documentclass mit: scrreprt, scrartcl, scrbook)
\pagestyle{headings} % schaltet Kopfzeilen ein
\clubpenalty = 10000 % schliesst Schusterjungen aus
\widowpenalty = 10000 % schliesst Hurenkinder aus

\usepackage{longtable} % erm�glicht die Verwendung von langen Tabellen
\usepackage{graphicx} % erm�glicht die Verwendung von Graphiken.
\usepackage{times}
\begin{document}

% B. TITELSEITE UND INHALTSVERZEICHNIS
% ***************************************************************************************************

\titlehead{Universit�t Koblenz-Landau\\
Institut f�r Softwaretechnik\\
Universit�tsstr. 1\\
56072 Koblenz}

\subject{Studienarbeit Java-Faktenextraktor f�r GUPRO}
\title{Ergebnisse der Besprechung vom 21.12.2005}
\author{Nicolas Vika \url{ultbreit@uni-koblenz.de}\\Arne Baldauf \url{abaldauf@uni-koblenz.de}}
\date{\today}
\maketitle
\newpage

%\tableofcontents
% Dieser Befehl erstellt das Inhaltsverzeichnis. Damit die Seitenzahlen korrekt sind, muss das Dokument zweimal gesetzt werden!
\newpage

% C. DOKUMENTHISTORIE
% ***************************************************************************************************
\begin{table}
	\begin{center}
	\begin{tabular}{|l|l|l|l|l|}
	  \hline
	  Version & Status & Datum & Autor(en) & Erl�uterung \\
	  \hline \hline
		1.0 & WIP & 22.12.2005 & Nicolas Vika & Initiale Version\\ \hline
		1.1 & WIP & 06.01.2006 & Arne Baldauf & Abnahme \& Erg�nzung\\ \hline
	\end{tabular}
	\end{center}
\end{table}

% D. HAUPTTEIL
% ***************************************************************************************************
\begin{itemize}
  \item{Der Faktenextraktor als PlugIn f�r Eclipse sollte der letzte Ausweg sein. Besser ist ein Kommandozeilenwerkzeug, welches autark arbeitet.}
  \item{Deshalb mehr Recherche im Bereich der Parser n�tig. Wichtig ist ein Werkzeug was einer "`verl�sslichen"' Quelle entstammt. D. h., dass es weiterhin gepflegt wird (im Hinblick auf neue Java Versionen - damit wir das in Zukunft nicht selbst machen m�ssen) und, dass es auch wirklich h�lt, was es verspricht. Tips:}
  \begin{itemize}
    \item{fujaba}
    \item{Suns Java Compiler oder den gcc "`anzapfen"'}
    \item{DMM, Dargstuhl, Lethbridge}
    \item{AST-Viewer (siehe Hinterw�ller-Dipl.-Arbeit)}
    \item{JRefactory}
    \item{JavaCC}
  \end{itemize}
  \item{Der Parser muss nicht unbedingt die aktuellste Java-Version unterst�tzen, es sollte aber nach M�glichkeit absehbar sein, dass dies in der Zukunft implementiert werden wird}
  \item{Wichtig ist genau zu wissen, was der Parser zur�ckgibt, um den n�chsten Schritt im Graphenaufbau zu kennen.}
  \item{Per Reflection an die Signaturen der class-Dateien herausziehen. Muss nicht unbedingt "`per Hand"' geschehen, vielleicht existieren daf�r schon Werkzeuge.}
  \item{Es reicht nur syntaktisch korrekte Programme verarbeiten zu k�nnen, eine gewisse Fehlertoleranz der Parser w�re aber w�nschenswert, da kein "`reales"' System ohne Fehler ist.}
  \item{Es reicht, wenn nur vorhandene (eigene) Java-Sourcen feingranular auf einen Graphen abgebildet werden. Ausgeschlossen davon sind externe (nicht eigene) Klassen, wie das JDK.}
  \item{Die Besonderheiten von Java beachten, da diese eventuell beim Parsen und der sp�teren Anfrageverarbeitung zum Tragen kommen (k�nnten). Beispiele:}
  \begin{itemize}
    \item{nested/inner classes (als Datei mit \$ davor)}
    \item{modifier protected (kann bei "`komischer"' Vererbung zu ungewollten Sichtbarkeiten f�hren)}
    \item{anonymous classes}
  \end{itemize}
  \item{Die Systemgr��e bedingt das Schema. Je gr��er das System, desto grober sollte das angewandte Schema f�r den Graphen sein, da sonst die Verarbeitungszeit zu hoch sein k�nnte.}
  \item{Mehr Dokumentation der Arbeiten. Auch Ergebnisse der Internetsuche, im Hinblick auf schriftlichen Teil der Studienarbeit (Bsp. warum taugt Werkzeug XYZ nicht).}
  \item{�berarbeitung der Anforderungsliste n�tig.}
  \item{Ergebnisse der Besprechungen zuk�nftig dokumentieren.}
  \item{Mehr Kommunikation zu Betreuern n�tig, offene Fragen ruhig direkt per E-Mail stellen.}
  \item{N�chstes Treffen am Donnerstag, dem 19.01.2006 um 16:00 Uhr in MB 108.}
  \item{Einreichen der Dokumente f�r das n�chste Treffen bis 17.01.2006 (Abends).}
\end{itemize}

\end{document}