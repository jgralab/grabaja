% Diese Vorlage wurde von Simon Berwert erstellt. Weitere Erkl�rungen findest du auf folgender Seite: http://www.unimac.ch/students/latex.de.html



% A. PR�AMBEL
% ***************************************************************************************************

\documentclass[smallheadings,headsepline, titlepage,12pt,a4paper]{scrartcl}
% Hier gibt man an, welche Art von Dokument man schreiben m�chte.
% Möglichkeiten in {}: scrartcl, scrreprt, scrbook, aber auch: article, report, book
\usepackage[ngerman]{babel} % erm�glicht deutsche Silbentrennung und direkte Eingabe von Umlauten, ...
\usepackage{ucs}
\usepackage[ansinew]{inputenc} % teilt LaTeX die Texcodierung mit. Bei Windowssystemen: ansinew
\usepackage[T1]{fontenc} % erm�glicht die Silbentrennung von Wörtern mit Umlauten
\usepackage{hyperref} % PDF wird mit Lesezeichen (verlinktes Inhaltsverzeichnis) versehen (bei Betrachtung mit Acrobat Reader sichtbar)
\typearea{12} % Breite des bedruckten Bereiches vergr�ssern (funktioniert nur in \documentclass mit: scrreprt, scrartcl, scrbook)
\pagestyle{headings} % schaltet Kopfzeilen ein
\clubpenalty = 10000 % schliesst Schusterjungen aus
\widowpenalty = 10000 % schliesst Hurenkinder aus

\usepackage{longtable} % erm�glicht die Verwendung von langen Tabellen
\usepackage{graphicx} % erm�glicht die Verwendung von Graphiken.
\usepackage{times}
\begin{document}

% B. TITELSEITE UND INHALTSVERZEICHNIS
% ***************************************************************************************************

\titlehead{Universit�t Koblenz-Landau\\
Institut f�r Softwaretechnik\\
Universit�tsstr. 1\\
56072 Koblenz}

\subject{Studienarbeit Java-Faktenextraktor f�r GUPRO}
\title{Aufgabenstellung}
\author{Arne Baldauf \url{abaldauf@uni-koblenz.de}\\ Nicolas Vika \url{ultbreit@uni-koblenz.de}}
\date{23. November 2005}
\maketitle
\newpage

%\tableofcontents
% Dieser Befehl erstellt das Inhaltsverzeichnis. Damit die Seitenzahlen korrekt sind, muss das Dokument zweimal gesetzt werden!
\newpage

% C. DOKUMENTHISTORIE
% ***************************************************************************************************
\begin{table}
	\begin{center}
	\begin{tabular}{|l|l|l|l|p{4,5cm}|}
	  \hline
	  Version & Status & Datum & Autor(en) & Erl�uterung \\
	  \hline \hline
		1.0 & fertig & 23.11.2005 & Arne Baldauf, Nicolas Vika & Initiale Version\\ \hline
		1.1 & fertig & 05.01.2005 & Nicolas Vika & Fehler berichtigt, in LaTeX �berf�hrt\\ \hline
	\end{tabular}
	\end{center}
\end{table}

% D. HAUPTTEIL
% ***************************************************************************************************
Ziel der Studienarbeit ist es einen Java-Faktenextraktor f�r Gupro zu erstellen (der die Sprachelemente bis Java 5 verarbeiten kann).\\

Dazu sollen sich die Studierenden selbst�ndig in das Thema GUPRO einarbeiten und einen L�sungsansatz entwickeln, der, soweit m�glich, bereits existierende L�sungen einbezieht (z. B. das Eclipse Java Schema, Parser etc.).\\

Die Implementierungssprache steht zur freien Wahl, sollte aber mit der Integration bereits existierender L�sungen harmonieren.
Alle Quelltexte sollen nach den bekannten Prinzipien aus der Vorlesung �Programmierung� erstellt werden.\\

Neben dem eigentlichen Faktenextraktor sollen auch mehrere Java-Schemata erstellt werden, die der Extraktor zur Berechnung des Graphen alternativ verwendet. Die Schemata sollen, neben einer feink�rnigen Graphenrepr�sentation des Codes, auch h�here Abstraktionsniveaus abdecken.\\

Der Graph soll ein TGraph sein (ein Zwischenschritt �ber GXL ist erlaubt, wenn dies den Aufwand der Studienarbeit sinnvoll reduziert).\\

F�r die Arbeit mit dem Graphen sollen auch einige sinnvolle GreQL-Anfragen formuliert werden, die insbesondere auf die objektorientierte Ausrichtung von Java abzielen.\\

Der Extraktor soll effizient arbeiten. Zur �berpr�fung werden (als Benchmark) Graphen aus Teilen des Java-SDKs extrahiert.\\

Das Gesamtarbeitspensum soll gleichm��ig auf beide Teilnehmer verteilt sein. Dazu sollen alle Arbeitsschritte von den Studierenden dokumentiert werden,  so dass ersichtlich ist, wer welche Arbeiten vorgenommen hat.\\

Zwecks Kontrolle, Absprache und R�ckmeldung soll regelm�ssig, alle zwei bis drei Wochen, ein Treffen der Teilnehmer mit den Betreuern stattfinden.\\

\end{document}