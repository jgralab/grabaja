% Diese Vorlage wurde von Simon Berwert erstellt. Weitere Erkl�rungen findest du auf folgender Seite: http://www.unimac.ch/students/latex.de.html



% A. PR�AMBEL
% ***************************************************************************************************

\documentclass[smallheadings,headsepline, titlepage,12pt,a4paper]{scrartcl}
% Hier gibt man an, welche Art von Dokument man schreiben m�chte.
% Möglichkeiten in {}: scrartcl, scrreprt, scrbook, aber auch: article, report, book
\usepackage[ngerman]{babel} % erm�glicht deutsche Silbentrennung und direkte Eingabe von Umlauten, ...
\usepackage{ucs}
\usepackage[ansinew]{inputenc} % teilt LaTeX die Texcodierung mit. Bei Windowssystemen: ansinew
\usepackage[T1]{fontenc} % erm�glicht die Silbentrennung von W�rtern mit Umlauten
\usepackage{hyperref} % PDF wird mit Lesezeichen (verlinktes Inhaltsverzeichnis) versehen (bei Betrachtung mit Acrobat Reader sichtbar)
\typearea{12} % Breite des bedruckten Bereiches vergr�ssern (funktioniert nur in \documentclass mit: scrreprt, scrartcl, scrbook)
\pagestyle{headings} % schaltet Kopfzeilen ein
\clubpenalty = 10000 % schliesst Schusterjungen aus
\widowpenalty = 10000 % schliesst Hurenkinder aus

\usepackage{longtable} % erm�glicht die Verwendung von langen Tabellen
\usepackage{graphicx} % erm�glicht die Verwendung von Graphiken.
\usepackage{times}
\begin{document}

% B. TITELSEITE UND INHALTSVERZEICHNIS
% ***************************************************************************************************

\titlehead{Universit�t Koblenz-Landau\\
Institut f�r Softwaretechnik\\
Universit�tsstr. 1\\
56072 Koblenz}

\subject{Studienarbeit Java-Faktenextraktor f�r GUPRO}
\title{Ergebnisse des Treffens vom 18. Januar 2007}
\author{Arne Baldauf \url{abaldauf@uni-koblenz.de}\\ Nicolas Vika \url{ultbreit@uni-koblenz.de}}
\date{\today}
\maketitle
\newpage

%\tableofcontents
% Dieser Befehl erstellt das Inhaltsverzeichnis. Damit die Seitenzahlen korrekt sind, muss das Dokument zweimal gesetzt werden!
%\newpage

% C. DOKUMENTHISTORIE
% ***************************************************************************************************
%\begin{table}
%	\begin{center}
%	\begin{tabular}{|l|l|l|l|l|}
%	  \hline
%	  Version & Status & Datum & Autor(en) & Erl�uterung \\
%	  \hline \hline
%		1.0 & WIP & 31.03.2006 & Arne Baldauf & initiale Version \\ \hline
%	\end{tabular}
%	\end{center}
%\end{table}

% D. HAUPTTEIL
% ***************************************************************************************************
\begin{itemize}
	\item{Es sollen folgende �nderungen am Text zum Metamodell durchgef�hrt werden:}
	\begin{itemize}
		\item{Korrekturen siehe handschriftliche Anmerkungen im Text.}
		\item{Die Einf�hrungen und Definitionen zu Modellen und Metamodellen sollen ausf�hrlicher und f�r Personen ohne GUPro-Kenntnisse verst�ndlich erweitert werden.}
		\item{Die verwendete Ich-/Wir-Form soll entfernt werden.}
		\item{Die Vermischung von Kapiteln und Versionen des Metamodells soll entfernt werden.}
		\item{Die Messung der Wirkung von Heuristiken soll in Anzahl der Klassen statt Diagrammgr��e in Seiten gemessen werden.}
		\item{Bei der "`finalize"'-Methode handelt es sich nicht um den Destruktor, sondern um eine von diesem aufgerufene Methode.}
		\item{Es soll hervorgehoben werden, da� es sich bei Kanten im Modell mit Multiplizit�ten > 1 um geordnete ("`ordered"') Kanten handelt, auch wenn diese im Modell nicht diesbez�glich markiert sind.}
	\end{itemize}
	\item{Es sollen folgende �nderungen am Metamodell durchgef�hrt werden:}
	\begin{itemize}
		\item{Die "`FoldGraph"'-Klasse soll in "`FoldGraphReference"' umbenannt werden.}
		\item{Attribute wie etwa "`isTrue"' sollen zu "`True"' umbenannt, da erstere Bezeichnung f�r die Get- / Set-Methode verwendet werden sollen.}
		\item{Die vorhandenen Enumerations-Attribute sollen auch als solche gekennzeichnet werden.}
	\end{itemize}
	\item{Aus dem Metamodell soll das entspr. EERL/GRAL-Schema erstellt werden.}
	\item{Aus dem o.g. Schema sollen mit GUPro die Klassen der Kanten und Knoten erzeugt werden.}
	\item{Der Funktionalit�t des Java-Extraktors soll soweit implementiert werden, dass eine Umwandlung von ANTLR-ASTs in TGraph-ASTs m�glich ist.}
  \item{Das n�chste Treffen findet am 07. Februar 2007 um 15:00s.t. in B108 statt (�nderung nach Absprache: Treffen nur mit Dr. Riediger am gleichen Termin in B121).}
  \item{Einreichung der Dokumente bis 05. Februar 2007 (abends).}
  \end{itemize}
\end{document}