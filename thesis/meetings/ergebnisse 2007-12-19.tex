% Diese Vorlage wurde von Simon Berwert erstellt. Weitere Erkl�rungen findest du auf folgender Seite: http://www.unimac.ch/students/latex.de.html



% A. PR�AMBEL
% ***************************************************************************************************

\documentclass[smallheadings,headsepline, titlepage,12pt,a4paper]{scrartcl}
% Hier gibt man an, welche Art von Dokument man schreiben m�chte.
% Möglichkeiten in {}: scrartcl, scrreprt, scrbook, aber auch: article, report, book
\usepackage[ngerman]{babel} % erm�glicht deutsche Silbentrennung und direkte Eingabe von Umlauten, ...
\usepackage{ucs}
\usepackage[ansinew]{inputenc} % teilt LaTeX die Texcodierung mit. Bei Windowssystemen: ansinew
\usepackage[T1]{fontenc} % erm�glicht die Silbentrennung von W�rtern mit Umlauten
\usepackage{hyperref} % PDF wird mit Lesezeichen (verlinktes Inhaltsverzeichnis) versehen (bei Betrachtung mit Acrobat Reader sichtbar)
\typearea{12} % Breite des bedruckten Bereiches vergr�ssern (funktioniert nur in \documentclass mit: scrreprt, scrartcl, scrbook)
\pagestyle{headings} % schaltet Kopfzeilen ein
\clubpenalty = 10000 % schliesst Schusterjungen aus
\widowpenalty = 10000 % schliesst Hurenkinder aus

\usepackage{longtable} % erm�glicht die Verwendung von langen Tabellen
\usepackage{graphicx} % erm�glicht die Verwendung von Graphiken.
\usepackage{times}
\begin{document}

% B. TITELSEITE UND INHALTSVERZEICHNIS
% ***************************************************************************************************

\titlehead{Universit�t Koblenz-Landau\\
Institut f�r Softwaretechnik\\
Universit�tsstr. 1\\
56072 Koblenz}

\subject{Studienarbeit Java-Faktenextraktor f�r GUPRO}
\title{Ergebnisse des Treffens vom 19. Dezember 2007}
\author{Arne Baldauf \url{abaldauf@uni-koblenz.de}\\ Nicolas Vika \url{ultbreit@uni-koblenz.de}}
\date{\today}
\maketitle
\newpage

%\tableofcontents
% Dieser Befehl erstellt das Inhaltsverzeichnis. Damit die Seitenzahlen korrekt sind, muss das Dokument zweimal gesetzt werden!
%\newpage

% C. DOKUMENTHISTORIE
% ***************************************************************************************************
%\begin{table}
%	\begin{center}
%	\begin{tabular}{|l|l|l|l|l|}
%	  \hline
%	  Version & Status & Datum & Autor(en) & Erl�uterung \\
%	  \hline \hline
%		1.0 & WIP & 01.06.2007 & Arne Baldauf & initiale Version \\ \hline
%	\end{tabular}
%	\end{center}
%\end{table}

% D. HAUPTTEIL
% ***************************************************************************************************
\begin{itemize}
	\item{Folgende �nderungen sollen am Extraktor vorgenommen werden:}
	\begin{itemize}
		\item{Statt Vector-Typen sollen ArrayList-Typen zum Einsatz kommen.}
		\item{Ausdr�cke wie \texttt{a.this} sollen aufgel�st werden (dies ist z.B. bei Zugriffen auf eine Instanz der Oberklasse m�glich).}
		\item{Die semantischen Kanten, welche beim Resolving erzeugt werden, sollen keine Attribute f�r Positionsinformationen besitzen:}
		\begin{itemize}
			\item{\texttt{IsTypeDefinitionOf}}
			\item{\texttt{IsDeclarationOfAccessedField}}
			\item{\texttt{IsDeclarationOfInvocatedMethod}}
		\end{itemize}
		\item{Identifier-Knoten sollen der Konsistenz wegen auch in Teilen des Graphs bestehen, welche mittels Reflection generiert wurden.}
		\item{Werden mittels Reflection neue Packages erkannt, so soll auch die zu den Packages geh�rigen Struktur erzeugt werden.}
		\item{Im Modus \texttt{COMPLETE} sollen auch alle Methoden und Felder per Reflection generiert werden.}
		\item{Felder und Methoden von Array-Typen m�ssen nicht aufgel�st werden (wie etwa \texttt{a.length} oder \texttt{a.concat()}).}
		\item{Mehrfach dem Extraktor �bergebene Dateien sollen nur einmalig geparst werden.}
		\item{Sind unter den �bergebenen Dateien / Verzeichnissen keinerlei geeignete Dateien, so soll auch kein Graph erzeugt werden.}
		\item{Nur aus Kommentaren bestehende Klassen sollen nicht mehr zu einem H�ngen des Extraktors f�hren.}
	\end{itemize}
  \item{Das n�chste Treffen findet am 10. Januar 2008 um 11:00 s.t. in B108 statt.}
  \item{Einreichung der Dokumente bis 09. Januar 2008 (morgens).}
  \end{itemize}
\end{document}