% Diese Vorlage wurde von Simon Berwert erstellt. Weitere Erkl�rungen findest du auf folgender Seite: http://www.unimac.ch/students/latex.de.html



% A. PR�AMBEL
% ***************************************************************************************************

\documentclass[smallheadings,headsepline, titlepage,12pt,a4paper]{scrartcl}
% Hier gibt man an, welche Art von Dokument man schreiben m�chte.
% Möglichkeiten in {}: scrartcl, scrreprt, scrbook, aber auch: article, report, book
\usepackage[ngerman]{babel} % erm�glicht deutsche Silbentrennung und direkte Eingabe von Umlauten, ...
\usepackage{ucs}
\usepackage[ansinew]{inputenc} % teilt LaTeX die Texcodierung mit. Bei Windowssystemen: ansinew
\usepackage[T1]{fontenc} % erm�glicht die Silbentrennung von Wörtern mit Umlauten
\usepackage{hyperref} % PDF wird mit Lesezeichen (verlinktes Inhaltsverzeichnis) versehen (bei Betrachtung mit Acrobat Reader sichtbar)
\typearea{12} % Breite des bedruckten Bereiches vergr�ssern (funktioniert nur in \documentclass mit: scrreprt, scrartcl, scrbook)
\pagestyle{headings} % schaltet Kopfzeilen ein
\clubpenalty = 10000 % schliesst Schusterjungen aus
\widowpenalty = 10000 % schliesst Hurenkinder aus

\usepackage{longtable} % erm�glicht die Verwendung von langen Tabellen
\usepackage{graphicx} % erm�glicht die Verwendung von Graphiken.
\usepackage{times}
\begin{document}

% B. TITELSEITE UND INHALTSVERZEICHNIS
% ***************************************************************************************************

\titlehead{Universit�t Koblenz-Landau\\
Institut f�r Softwaretechnik\\
Universit�tsstr. 1\\
56072 Koblenz}

\subject{Studienarbeit Java-Faktenextraktor f�r GUPRO}
\title{Ergebnisse des Treffens vom 19. Januar 2006}
\author{Arne Baldauf \url{abaldauf@uni-koblenz.de}\\ Nicolas Vika \url{ultbreit@uni-koblenz.de}}
\date{\today}
\maketitle
\newpage

%\tableofcontents
% Dieser Befehl erstellt das Inhaltsverzeichnis. Damit die Seitenzahlen korrekt sind, muss das Dokument zweimal gesetzt werden!
\newpage

% C. DOKUMENTHISTORIE
% ***************************************************************************************************
\begin{table}
	\begin{center}
	\begin{tabular}{|l|l|l|l|l|}
	  \hline
	  Version & Status & Datum & Autor(en) & Erl�uterung \\
	  \hline \hline
		1.0 & WIP & 19.01.2006 & Arne Baldauf & initiale Version \\ \hline
	\end{tabular}
	\end{center}
\end{table}

% D. HAUPTTEIL
% ***************************************************************************************************
\begin{itemize}
  \item{Das Dokument zur Online-Recherche soll �berarbeitet werden, insbesondere soll eine gemeinsame Bewertungsgrundlage der Tools ersichtlich werden, Ausschlusskriterien sollen explizit aufgef�hrt werden}
  \item{Das Werkzeug ANTLR soll im genaueren untersucht werden:}
  \begin{itemize}
  	\item{Werden Kommentare mit in den AST aufgenommen, oder zeigt lediglich die graph. Ausgabekomponente diese nicht an?}
  	\item{Sind Positionsangaben der jeweiligen Elemente mit im AST gespeichert?}
  	\item{Welche Ursachen bedingen im AST vorhandene leere "`extends"'- oder "`implements"'-Elemente?}
  	\item{Wie reagiert der Parser auf Besonderheiten (wie etwa fehlende Carriage Returns am Dokumentenende, Kommentare mitten in Statements, unterschiedliche Zeilenumbr�che innerhalb einer Datei, Blanks am Zeilenende(hinter Semikolon), Tabulatoren, etc.), welche die typischen Kandidaten f�r fehlerhafte Positionsangaben sind (nur falls auch wirklich Positionsangaben in der internen Darstellung vorhanden sind)?}
  	\item{Probeweises parsen eines JDK-Packages (evtl. auch JDK komplett?) hinsichtlich Stabilit�t und Geschwindigkeit}
  	\item{Untersuchung der Dokumentation hinsichtlich einer API o.�.}
  	\item{Untersuchung der Dokumentation hinsichtlich der Funktionalit�ten der Grammatiken (da der Parser LL-basiert arbeitet, die Grammatiken aber in EBNF vorliegen}
  	\item{Abschlie�end: Festlegung der notwendigen Schritte und Absch�tzung des Aufwands daf�r}
  	\end{itemize}
  \item{Untersuchung des Modells, welches unter Eclipse liegt (EMF - Eclipse Modeling Framework), auf Funktionsweise und Integrierbarkeit in unsere Aufgabenstellung, da dieses definitiv die Kontextinformationen und Positionsangaben beinhaltet, die Semantik erkennt und Referenzen aufl�sen kann}
  \item{Untersuchung einer quelloffenen Referenzimplementation eines Java-Compilers, z.B. des GCJ, hinsichtlich dessen interner Repr�sentation des AST}
  \item{Kontakt mit Lethbridge aufnehmen, ob bei seiner L�sung ein Parser verwendet wird}
  \item{Untersuchung von JavaCUP auf dessen interne Repr�sentation / Vorhandensein eines AST}
  \item{�berarbeitung der Anforderungsliste an einigen kleineren Stellen}
  \item{N�chstes Treffen findet am 06. Februar 2006 um 14 Uhr in MB108 statt.}
  \item{Einreichung der Dokumente bis 02. Februar 2006 (abends)}
  \end{itemize}
\end{document}