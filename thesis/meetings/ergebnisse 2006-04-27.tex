% Diese Vorlage wurde von Simon Berwert erstellt. Weitere Erkl�rungen findest du auf folgender Seite: http://www.unimac.ch/students/latex.de.html



% A. PR�AMBEL
% ***************************************************************************************************

\documentclass[smallheadings,headsepline, titlepage,12pt,a4paper]{scrartcl}
% Hier gibt man an, welche Art von Dokument man schreiben m�chte.
% Möglichkeiten in {}: scrartcl, scrreprt, scrbook, aber auch: article, report, book
\usepackage[ngerman]{babel} % erm�glicht deutsche Silbentrennung und direkte Eingabe von Umlauten, ...
\usepackage{ucs}
\usepackage[ansinew]{inputenc} % teilt LaTeX die Texcodierung mit. Bei Windowssystemen: ansinew
\usepackage[T1]{fontenc} % erm�glicht die Silbentrennung von W�rtern mit Umlauten
\usepackage{hyperref} % PDF wird mit Lesezeichen (verlinktes Inhaltsverzeichnis) versehen (bei Betrachtung mit Acrobat Reader sichtbar)
\typearea{12} % Breite des bedruckten Bereiches vergr�ssern (funktioniert nur in \documentclass mit: scrreprt, scrartcl, scrbook)
\pagestyle{headings} % schaltet Kopfzeilen ein
\clubpenalty = 10000 % schliesst Schusterjungen aus
\widowpenalty = 10000 % schliesst Hurenkinder aus

\usepackage{longtable} % erm�glicht die Verwendung von langen Tabellen
\usepackage{graphicx} % erm�glicht die Verwendung von Graphiken.
\usepackage{times}
\begin{document}

% B. TITELSEITE UND INHALTSVERZEICHNIS
% ***************************************************************************************************

\titlehead{Universit�t Koblenz-Landau\\
Institut f�r Softwaretechnik\\
Universit�tsstr. 1\\
56072 Koblenz}

\subject{Studienarbeit Java-Faktenextraktor f�r GUPRO}
\title{Ergebnisse des Treffens vom 27. M�rz 2006}
\author{Arne Baldauf \url{abaldauf@uni-koblenz.de}\\ Nicolas Vika \url{ultbreit@uni-koblenz.de}}
\date{\today}
\maketitle
\newpage

%\tableofcontents
% Dieser Befehl erstellt das Inhaltsverzeichnis. Damit die Seitenzahlen korrekt sind, muss das Dokument zweimal gesetzt werden!
\newpage

% C. DOKUMENTHISTORIE
% ***************************************************************************************************
\begin{table}
	\begin{center}
	\begin{tabular}{|l|l|l|l|l|}
	  \hline
	  Version & Status & Datum & Autor(en) & Erl�uterung \\
	  \hline \hline
		1.0 & WIP & 31.03.2006 & Arne Baldauf & initiale Version \\ \hline
	\end{tabular}
	\end{center}
\end{table}

% D. HAUPTTEIL
% ***************************************************************************************************
\begin{itemize}
	\item{Es soll ein Metamodell f�r die feingranulare Abbildung von Java-Quelltexten erstellt werden:}
	\begin{itemize}
		\item{�bernehmen von fremden Modellen nur insoweit, wie die Quelle den eigenen Aufwand mindert.}
		\item{�bernehmen von fremden Modellen nur insoweit, wie die Quelle unserer eigenen Struktur entspricht oder ohne gr��eren Aufwand daran angepasst werden kann.}
	\end{itemize}
	\item{�berarbeitung des Dokumentes antlr.tex:}
	\begin{itemize}
		\item{Eigener Darstellungsstil f�r Grammatiken, welcher sich von Codedarstellungen unterscheidet.}
		\item{Anderer Font anstelle von Hervorhebung im gleichen Font.}
		\item{Kleinere Schriftgr��e bei der Zeilennummerierung der Listings.}
		\item{Das verwendete Beispeil zu Anfang einf�hren und die Beispiele durchg�ngig darauf beziehen.}
		\item{Abschnitt 2.6: Das Beispiel aus 2.5 verwenden und Input/Output auflisten.}
		\item{Abschnitt 3.1: Alternative f�r Erzeugung des Offset (Sammeln der Informationen in einem zweiten Durchlauf) nennen.}
		\item{Abschnitt Speicherung des Offset: Die beiden Alternativen vorher nennen und Vorgehen aufzeigen.}
		\item{Abschnitt 3: Klassendiagramm und Aktivit�tsdiagramm von einander trennen.}
	\end{itemize}
	\item{Inhaltliche Erg�nzungen und �berarbeitungen:}
	\begin{itemize}
		\item{Das Vorkommen einer einzelnen Raute (\#) in der Grammatik bedingt einen Abstieg im AST}
	\end{itemize}
	\item{Kl�rung einiger offenstehender Fragen:}
	\begin{itemize}
		\item{Ist die L�ngeninterpretation von Escapesequenzen bei der Abfrage von Token.Text.Length richtig?}
		\item{Besitzen innere Knoten eine Tokenreferenz oder ist dies lediglich bei den Bl�ttern der Fall?}
		\item{Zu welchem Zeitpunkt werden die semantischen Aktionen ausgef�hrt (z.B. bei Kleenee-*)?}
	\end{itemize}
  \item{Das n�chste Treffen findet am 16. Mai 2006 um 16 Uhr in MB108 statt.}
  \item{Einreichung der Dokumente bis 14. Mai 2006 (abends).}
  \end{itemize}
\end{document}